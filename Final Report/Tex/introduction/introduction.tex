
\chapter{Introduction}

\section{Motivation and Objectives}

Despite the high uptake of technology through the university education system, one aspect remains very manual: Exam Invigilation.

During each exam session, it is common to have invigilators spread across multiple rooms, either due to the popularity of certain courses, timetable collisions, or having students with extra time.

There are many reasons for which invigilators may need to communicate during these sessions.  For example, examination start and finish times need to be coordinated, clarifications or corrections need sharing amongst exam rooms, and students may need escorting to the bathroom.  This can be particularly troublesome for rooms with single invigilators.

Whilst most invigilators carry a mobile phone which could be used for communicating with other rooms, this is not a reliable solution to the communication issue.  Not all invigilators do carry a phone, and invigilators do not necessarily know the number of all other invigilators in other rooms.  Furthermore, it is necessary for all mobile phones to be placed in silent mode, leading to communications being missed.

Improving ease and speed of communication between exam invigilators is highly desirable as, ultimately, it will improve the quality of the student exam experience.  Quick and efficient communication will mean problems can be identified, shared and resolved more quickly, meaning students can receive corrections, visit the bathroom or obtain more paper sooner.  This allows the students to spend their time focussed on the task at hand; answering the paper in front of them.

The project is to create an easy to use and effective real-time communication system capable of providing the necessary functionality to enable invigilators to successfully run exams, and simplify the process for all involved.
