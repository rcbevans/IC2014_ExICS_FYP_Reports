\chapter{System Design and Architecture}

\label{ch:systemarchitecture}

This chapter will give an overview of the system architecture chosen to implement ExICS (Exam Invigilator Communication System) and the choices made as to which technologies would be employed to create each part of the system and the reasoning.

\section{Consideration of System Network Architecture}

\FloatBarrier

With any distributed system offering communication between client applications it is necessary to choose a system network architecture to facilitate the communication.

Distributed system can be physically instructed by two ways: First, fully connected network peer-to–peer approach in which each of the nodes is connected to each other. Second, partially connected network in which a direct links exist between some but not all pairs of computers. A few of the partially connected network models are star (client server) structured networks, multi-access bus networks ring structured networks, and tree-structured networks.\cite{designingChatRoomApp}

The two models considered for ExICS were star (client-server) and fully connected (peer-to-peer) networks, shown in figure \ref{fig:network_models}.  With each network communication model there are pros and cons which it was important to carefully consider before beginning design of the system.

Figure \ref{subfig:client-server} shows a representation of the client server model where all clients connect to only a centralised server to communicate.  In this case, the server is responsible for listening to messages from connected clients and forwarding the messages to other connected clients based upon the contents of the received message.

In a peer-to-peer network, shown in figure \ref{subfig:peer-to-peer}, all clients responsible for discovering and maintaining connection to the other clients in the local network and so have the ability to directly communicate with the intended recipient for any message.

\begin{figure}[!htbp]
	\centering
	\begin{subfigure}[b]{0.4\textwidth}
		\includegraphics[width=\textwidth]{"System Diagram/client_server_simpler"}
		\caption{Client-Server Model}
		\label{subfig:client-server}
	\end{subfigure}
	\begin{subfigure}[b]{0.4\textwidth}
		\includegraphics[width=\textwidth]{"System Diagram/peer_to_peer"}
		\caption{Peer-to-Peer Model}
		\label{subfig:peer-to-peer}
	\end{subfigure}
	\caption{Comparison of Client-Server and Peer-to-Peer Network Models}
	\label{fig:network_models}
\end{figure}

Use of the client-server model negates some of the complexities involved with peer-to-peer; discovering and maintaining the peer network.  By having a static central server, any clients wishing to connect can negotiate opening a connection with the static IP address, after which any messages for other clients can be routed by the central server which has access to all connections of peers.  This does however introduce a single point of failure;  if the central server becomes unavailable, no clients are able to communication, rendering the system useless.  There are, however, means to negate this threat such as running fallback servers which can be used should the main server become unavailable.

A benefit of the peer-to-peer networking model is efficiency and network utilisation.  When a message needs to be sent to a peer, the packet can be directly router to that client, rather than via the central server which is then required to find the client connection itself whilst dealing with all other system traffic being routed through it.

After consideration of the pros and cons of each approach the decision was made to use a client-server model in the system design.  Client-server architecture was chosen for a number of reasons:

\begin{itemize}

\item Firstly, simplicity.  With data that needs pulling in from back-end servers in DoC and synchronisation of system state, use of a central server means that the responsibility for such complicated actions can lie in one place.  The central server can, when necessary, do the system state synchronisation and pull in the required state data from the APIs providing it, update its own internal system state, and push this to connected clients.

\item Secondly, coherency.  With the system state being controlled and maintained by the central server, it is possible to implement a request and response system with client instigated changes of system state.  If a client, for example, wishes to pause an exam at the same time as another client wishes to finish it, the central server can service the requests as they arrive, updating its own central state and pushing the new state to connected clients.  As a result, it is guaranteed that all clients will always be exposed to the same system state.  In a peer to peer network however, both clients could broadcast their state change information simultaneously, and those receiving the message could attempt to apply the changes in a different order leading to different local states for each client.

\item Finally, reliability.  With a central server in control of the state of the system and responsible for making changes to that state using a request response method, it is possible for the server to log server state, changes as they are made and who requested them.  With this, it should be possible to recreate any system state from past logs.  This is beneficial in a case where the server crashes unexpectedly, in which case system state can be recreated as it was so no state information is lost.  Using a peer-to-peer network, however, if all clients were to crash, close, or disconnect mid-session, it would be far more difficult to recreate the system state and arbitrate which state is correct if multiple clients were to have differing versions.

\end{itemize}

\FloatBarrier

\section{System Diagram}

\FloatBarrier

Figure \ref{fig:system_diagram} shows the client-server system architecture, discussed in the previous section, as used in ExICS.  It makes use of existing college and departmental infrastucture to simplify the system implementation and integrate with existing data sources available within the department.

\begin{figure}[!htbp]
	\centering
	\includegraphics[width=0.9\textwidth]{"System Diagram/exics_system_diagram"}
	\caption{System Diagram}
	\label{fig:system_diagram}
\end{figure}

As mentioned in the background section of this report, the College offers an authentication system in the form of both Kerberos and LDAP.  It also offers a firewall protecting internal traffic from outsider snooping of traffic, as well as allowing those outside the College network to connect securely through VPN.  These existing tools are used to simplify security considerations of the system, allowing the focus of development to be on functionality and user experience.

The Department of Computing departmental servers also hold the exam schedule, seating plan and other information required to power ExICS.  These existing data sets and APIs exposing information are used to simplify integration of the system such that current means of storing and sharing exam schedule information do not need to change.  The implementation of the Exam Data API Wrapper server is discussed in chapter \ref{ch:apiwrappers}.

\FloatBarrier

\section{Available Implementation Technologies}

With the design of the system components decided, it was then necessary to investigate, consider and choose the technologies which would be employed to build the system.

\subsection{Network Communication Channels}

The most fundemental part of ExICS is the communication channel used for client-server communication.  It is the fabric that enables messages to be shared and system state to be updated.

For this purpose two potential options were considered, AJAX and Websockets, with Websockets being chosen as most suitable for the needs of the system from both a functionality and performance perspective.

WebSocket was developed as part of the HTML5 initiative and introduced the WebSocket JavaScript interface, defining a full-duplex single socket connection over which messages can be sent between client and server.

HTTP was designed as a request/response protocol bringing great complexity in cases where bi-directional communication was required.  The protocol was designed such that a client sends a request to the server and when and only when the request has been completely received, the server sends its response. As a result there are many techniques, or some may say 'hacks'\cite{httpabuse}, which have been employed to provide the illusion of full-duplex communication between client and server such as long-polling using AJAX\cite{lpAjax}.

WebSocket is purposefully designed to provide the functionality designed by modern real-time web applications and so made perfect sense for use in this project.
% 
% By design, WebSocket is designed with a subscription model.  Any clients connecting subscribe to the Socket and the WebSocket server simply re-emits any received messages to all users.  This is not optimal behaviour though as in some cases, such as when a message has a specific intended recipient, bandwidth is wasted in transmitting to all users.  This is not disasterous as it is possible to simply process the received message payload on the client device and establish in the client-side software if the current device is the intended recipient.

\subsection{Back-End Servers}
\label{subs:exics_backend_server}

There are a number of mature web socket platforms which are used in a wide range of scenarios which could have been suitable for use in ExICS.

Having experience with Apache servers running PHP during my six month placement, my first instinct was to consider the platform for use in the back end server.  I began researching Apache and PHP for use with websockets and found that whilst there were a number of sources of tutorials and information on using websockets with PHP a great many recommended other technologies to implement the server with better support, functionality and more geared towards persistant connections; which PHP is not.

While searching for alternatives, I found a large number of websocket implementations which were highly recommended, in a large number of languages ranging from Python and Ruby, to Java and Javascript.

One such platform very highly recommended for scenarios such as that of ExICS \cite{whyNodeJS} is Node.JS, a platform built on Chrome's Javascript runtime, designed for building fast, scalable network applications.  It uses an event driven non-blocking I/O model making it lightweight and efficient and perfect for data-intensive real time applications running across distributed devices.\cite{nodeJS}

One of the best features of Node.JS making it perfectly suited to the problem being tackled with ExICS and its client-server design where the server is responsible for maintaining system state is it's execution model.  Node.js, being implemented entirely in Javascript, makes use of it's single-threaded event-loop design meaning the ``app'', the main server code, is only ever executed on a single thread.\cite{understandingNodeEventLoop}  In simple terms, it makes all data structures, such as exam system state, completely safe.  Unlike Apache which spawns a new thread with every connection, making any shared variables such as system state inherently dangerous, adding implementation complexity.

Whilst a single thread of execution is great for ensuring any shared variables are safe to access at any point, questions may be raised as to whether the performance of such a system would be sufficient to cope as the system is scaled.  At this point it was necessary to consider ExICS and its nature.  In reality, each server will serve one department or team that need to communicate; this means that, despite the potential for tens of exams to be running during a session, there is limited space and so a limited number of exams that can run concurrently.  This limits the number of concurrent users to be proportional to this limited number of concurrent exams.  Different departments will use different ExICS server hosts, keeping the load to the system sufficiently small such that the single threaded execution won't pose an issue to performance, and won't require a supercomputer to run it.

Furthermore, the biggest performance killer in systems is Disk I/O.\cite{IOperfKiller}  For example, sending 2K bytes over 1 Gbps network takes approximately 20,000 ns, compared to 20,000,000 ns to read 1MB of sequential data from a disk, 1000x slower.\cite{diskIOlatency}  ExICS however is primarily concerned with system state held in local memory with very small amounts of disk I/O. Even network communication itself is slow, however, whilst in Node.JS, the user ``app'' code is executed on a single thread, the single threaded user code can dispatch tasks, such as listening on a port or to a socket, or writing data to a disk to Node.  In these cases, separate worker threads will go and complete the task, using callbacks to further execution of the ``app'' code when complete.  This means that even when a long, slow task needs completing, such as writing an event to the log file, after the task has been scheduled, the Node main thread can return to servicing other queued event callback execution, returning to the completed task when the callback is made.  

A downside to this approach is that Node makes no guarantees about when or in which order events will be processed, necessitating some error checking in the server code, for example checking that an exam has already been paused before allowing the system to resume it.  This is, however, far less of an issue or inconvenience than trying to implement the system using a truly multithreaded platform such as Apache.

Node also has a incredibly diverse range of libraries available through NPM (Node Package Manager), bringing support for a huge range of additional technologies such as Websocket servers and LDAP connectors, is stable and mature with a large amount of reference material, and being written in Javascript, is incredible easy to work with and fast to develop.

With this considered and having followed a number of Node.JS tutorials, I decided Node.JS offered a solid, easy to use and well-supported platform to work with.

With this I began to investigate Websocket libraries to use with Node.JS to create the websocket server to begin building from.  There were many available throught the NPM such as Websocket, Socket.IO and ws, all of which pass the extensive Autobahn Websockets conformance tests and offered both server and client support for testing purposes.

I scoured web resources, blogs and reference sites for recommendations and advice \cite{whichwebsocket} on choosing a websockets implementation to run with and found it a very difficult task.  All of the main websocket libraries I looked at seemed to have very positive reviews.

The implementation which seemed most highly rated in terms of ease to use was Socket.IO.  Unlike most other Node.JS websocket implementations which offer only server side websockets, Socket.IO offers libraries for both server and client side use.  It can transparently select the most suitable transport mechanism, even amongst concurrently connecting clients, meaning it works on any platform or browser.  It is more than just a low-level websockets implementation, it offers a layer of abstraction above the websocket protocol, simplifying the development process for those using it.  Rather than registering events for data received, as with basic websockets, Socket.IO allows users to transparently send and receive messages assigned to ``topics'', a string tag which means messages sent with the tag will trigger the event assigned to it on the other end of the protocol.

This would have been ideal for use in ExICS, as the nature of messages sent in the system means they fall into distinct topics, for example, start exam, stop exam, etc.  Unfortunately, whilst mobile libraries for Java and so Android now exist which support Socket.IO \cite{AndroidAsync}, at the time of investigating and choosing which technologies to develop ExICS with, I was unable to find a library supporting Socket.IO.  As a result I was forced to choose a more low-level websockets implementation.

Unable to use Socket.IO, I returned to the more basic pure websocket implementations, ``websocket'' and ``ws''.  With both libraries claiming full specification compliance and passing all Autobahn conformance tests, I decided to experiment with both.  I created a simple echo server using each library to look at ease of use and functionality.  Both libraries were very easy to work with and soon I had what I believed to be a functional echo server.

To test compatibility I used two tools, a Google Chrome WebApp ``Dark Websocket Terminal''\cite{dwst}, and wscat, a utility which comes with the ws Node library.  I attempted to connect to both echo servers using each tool to verify compatibility and test correctness.  I immediately found an issue, the ``Dark Websocket Terminal'' was unable to connect to the echo server implemented using the ``websocket'' library.

With the ``ws'' library working as expected, easy to use and highly recommended, I decided that it would be a good choice moving forward.

\subsection{Persistant Data Storage}

After deciding that I would proceed with developing using Node.JS for server side development, I needed to investigate a database which could be used for storing persistent data in the system, for example the details of exams which are logged after completing, or student registers taken by invigilators and notes made during examinations.

NPM gives access to an incredible range of database drivers for both relational structured database engines such as mySQL or postgresSQL, as well as non-relational noSQL such as mongoDB or couchDB.

Unfamiliar with the differences between SQL and noSQL database engines, I soon found that noSQL database engines such as mongoDB and couchDB are highly praised and recommended by those who work with node on a regular basis, judging by the answers and opinions of those on StackOverflow, however they stressed that the choice is a matter of preference and, whilst there may be a \textit{wrong} choice, there is often no \textit{right} one.\cite{databaseChoice}

With this in mind I looked into a number of the database drivers for Node, compared the style of writing the queries and the functionality they offered and decided I would progress making use of MongoDB thanks to its mature community and documentation, extensive functionality and incredible ease of use.

MongoDB stores its data in ``collections'' of documents, rather than in tables of data, playing incredibly nicely with Javascript as it means javascript objects, such as an exam object can be directly stored, with key value pairs of properties being searchable as well as indexable.  MongoDB also offers a RESTful API allowing documents stored to be manipulated directly using HTTP from remote clients.  This would allow any data created, stored and manipulated by ExICS to be retrievable from other software wishing to access it.

\subsection{System Security}

As mentioned in the background, College ICT offers three means of authorizing and authenticating members of the College; Kerberos, LDAP (Lightweight Directory Access Protocol) and directly using Microsoft Active Directory.

Kerberos is a network authentication protocol, designed to provide strong authentication for client-server applications using cryptography.\cite{whatiskerberos}  Kerberos makes use of tickets which are issued by a trusted ticket granting server, in this case, College ICT server, which can be passed to subsequent services such as new servers or webpages within a kerberos realm and verified by the new service to be valid, thus authenticating the user as someone authrorised to access the resource.\cite{explainlikeIm5}

A benefit of kerberos is that College credentials need only be shared with a trusted verifiable server; the kerberos server, not with ExICS itself.  This would simplify ExICS as all that needs to be implemented in the server is verification of authorization tickets received from connected clients referred on from the kerberos ticket server.

Using kerberos, however, would require time and effort on the part of the College IT department to configure test and verify the system before it would be usable with ExICS.

LDAP, in contrast is a protocol giving the means to query directory servers, a type of heirarchical NoSQL storage where data for individuals in an organisation are stored in a tree structure as an object containing a collection of name value pairs.

With LDAP and Active Directory, you are able to perform a number of actions such as verifying user credentials, querying user groups and other attributes belonging to that user.

Having investigated all three possibilities, I found that Node.JS provided libraries offering full functionality for all three means of authentication.  With Kerberos it would have complicated the client application with additional code necessary to negotiate and obtain authorization tickets and so decided that it would be best for my needs to have ExICS handle authentication of connecting users internally using the college LDAP or Active Directory services available.

Looking through the package listings available through NPM I found activedirectory, an ldapjs client for authN (authentication) and authZ (authorization) for Microsoft Active Directory with range retrieval support for large Active Directory installations.\cite{activedirectory}  This looked like it was easy to use and offered all the functionality necessary for authenticating users with good documentation available.

\subsection{Client Application Platform}

As discussed in the background of this report, it was decided during the initial investigation into the requirements and specification of the project that Android would be the chosen platform for the (initial) development of the ExICS client to avoid some of the pitfalls of other implementations such as a web apps being hidden behind the invigilators other work while invigilating, and thanks to the high market penetration of the Android Operating System.

Whilst researching libraries and resources that would be of use in developing the mobile client application I came across Xamarin, a development platform allowing core application code written in C\# to run across a wide range of platforms, including Android, iOS and Windows Phone, as well as Mac OSX and Windows.\cite{xamarin}  Xamarin allows native platform user interface layers to be built upon shared C\# user interface code, built upon shared C\# app logic, giving native performance of the application from a common code base.  It allows developers to focus on user experience, not worry about constantly reimplementing the code powering the UI underneath.  Furthermore, with Xamarin, support libraries written in Java, objective C, or C\# can be linked and used seamlessly, giving the best of all worlds.

Interested in the possibility of not only being able to offer an Android client application, but and iOS, I downloaded Xamarin and looked at some tutorials on how to get started.  I soon found that licensing was going to be an issue and that without an iOS device or any developer licenses from Apple, working on an iOS project was going to be virtually impossible.  With very little experience working with C\# I also found the change in language and ways of doing things challenging.  Despite being in regular contact with Xamarin support engineers in San Francisco, after a period spent evaluating the platform, I decided that, given the relatively short duration of the project and the amount there was to do, that I would be safer sticking with just Android; a platform with which I had past experience using Java, a language with which I was more familiar.

It was therefore necessary to find a suitable websockets library for use in the Android client application.  Whilst there are a number of Websocket libraries written in Java which may have been suitable for my purposes, one which was frequently mentioned as being very compliant with Websocket standards, and easy to work with was Autobahn Android, written and maintained by Autobahn, the company behind the Autobahn Websocket compliance test suite.